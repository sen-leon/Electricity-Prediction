\documentclass[11pt,table]{article}

\usepackage{subfiles}
\usepackage[breakable]{tcolorbox}
%\usepackage{parskip} 

\usepackage{iftex}
\ifPDFTeX
\usepackage[T1]{fontenc}
\usepackage{mathpazo}
\else
\usepackage{fontspec}
\fi

\usepackage{lipsum}

\usepackage{float}
\usepackage{caption}


\usepackage{subcaption}
\usepackage[skip=50pt]{caption} % default 10pt
\captionsetup[figure]{skip=10pt}
\usepackage{rotating}
\usepackage{wrapfig}

% could uncomment this to spare author typing "Figures" all the time
%\graphicspath{{Figures/}}

\usepackage{graphicx}

\captionsetup{format=plain,
	aboveskip=-0.2cm,
	indention=0cm, 
	textformat=simple,
	textfont=small,
	labelfont=small, 
	justification=centering,
	labelfont=bf}



\usepackage[Export]{adjustbox} % Used to constrain images to a maximum size
\adjustboxset{max size={0.9\linewidth}{0.9\paperheight}}
\usepackage{float}
%\floatplacement{figure}{H} % forces figures to be placed at the correct location
\usepackage{xcolor}    % Allow colors to be defined
\usepackage{enumerate} % Needed for markdown enumerations to work
\usepackage{geometry}  % Used to adjust the document margins
\usepackage{amsmath}   % math symbols
\usepackage{amssymb}   % more math symbols
\usepackage{textcomp}  % defines textquotesingle

\usepackage{upquote}   % Upright quotes for verbatim code
\usepackage{eurosym}   % defines \euro
\usepackage[mathletters]{ucs} % Extended unicode (utf-8) support
\usepackage{fancyvrb}  % verbatim replacement that allows latex
\usepackage{grffile}   % extends the file name processing of package graphics 
% to support a larger range
\makeatletter % fix for grffile with XeLaTeX
\def\Gread@@xetex#1{%
	\IfFileExists{"\Gin@base".bb}%
	{\Gread@eps{\Gin@base.bb}}%
	{\Gread@@xetex@aux#1}%
}
\makeatother

\usepackage{hyperref}
\usepackage[capitalise]{cleveref}
\Crefname{figure}{Fig.}{Figs.}% {<type>}{<singular>}{<plural>}



% The default LaTeX title has an obnoxious amount of whitespace. By default,
% titling removes some of it. It also provides customization options.
\usepackage{titling}
\usepackage{longtable} 
\usepackage{booktabs}  
\renewcommand{\arraystretch}{1.2} % more space between table rows

\usepackage[inline]{enumitem}
\usepackage[normalem]{ulem}
% normalem makes italics be italics, not underlines
\usepackage{mathrsfs}

\captionsetup[table]{skip=10pt}


% Colors for the hyperref package
\definecolor{urlcolor}{rgb}{0,.145,.698}
\definecolor{linkcolor}{rgb}{0,0,0}%{.71,0.21,0.01}
\definecolor{citecolor}{rgb}{0,0,0}%{.12,.54,.11}


\title{Machine Learning Methods in Mechanics Report SS21}
\author{Students}
\date{\today}



% Prevent overflowing lines due to hard-to-break entities
\sloppy 

\hypersetup{
	breaklinks=true,  % so long urls are correctly broken across lines
	colorlinks=true,
	urlcolor=urlcolor,
	linkcolor=linkcolor,
	citecolor=citecolor,
}
% Slightly bigger margins than the latex defaults

\geometry{verbose,tmargin=1in,bmargin=1in,lmargin=1in,rmargin=1in}

\newcommand{\ket}[1]{\left|{#1}\right\rangle}
\newcommand{\bra}[1]{\left\langle{#1}\right|}
\newcommand{\braket}[2]{\left\langle{#1}\middle|{#2}\right\rangle}



\usepackage{fancyhdr}
\addtolength{\headheight}{1.2cm} % make more space for the header
\pagestyle{fancy} %plain} % use fancy for all pages except chapter start
\fancyhead[L]{\leftmark}
%\lhead{\includegraphics[height=1.3cm]{logo2}} % left logo
\rhead{\includegraphics[height=0.6cm]{Figures/header.png}}   % right logo
%\renewcommand{\headrulewidth}{0pt} % remove rule below header


\title{Project Title}
\author{Luis Gentner, Leon Sengün, Dilara Yildiz}
\date{\today}



\begin{document}
\begin{titlepage} 
	\centering 
	\rule{\textwidth}{1pt} 
	\vspace{2pt}\vspace{-\baselineskip} 
	\rule{\textwidth}{0.4pt} 
	\vspace{0.1\textheight} 
	
	
	%%% Adjust your project title here
	{\Huge PREDICTING RENEWABLE ENERGY }\\[0.5\baselineskip] 
	{\Huge PRODUCTION USING }\\[0.5\baselineskip]
	%{\Large }\\[0.5\baselineskip] 
	{\Huge MACHINE LEARNING METHODS} 
	
	
	\vspace{0.025\textheight} 
	\rule{0.3\textwidth}{0.4pt} 
	\vspace{0.1\textheight}
	
	%%% Your names, if long, a "\\" in between may help to make things look better
	{\Large \textsc{Luis Gentner, Leon Sengün, Dilara Yildiz}} 
	
	\vfill 
	
	%%% You can include a nice image from your project here
	\includegraphics[width=0.5\linewidth]{Figures/example_cover.png} \\
	\vspace{0.05\textheight}
	{\large\textsc{Machine Learning Methods in Mechanics \\Report}\\ -\\ University of Stuttgart} 
	
	
	\vspace{0.1\textheight} 
	
	%%% Adjust the date here if you like
	{\normalsize \today}
	
	
	\rule{\textwidth}{0.4pt}
	\vspace{2pt}\vspace{-\baselineskip}
	\rule{\textwidth}{1pt}
	
\end{titlepage}

\pagenumbering{roman}

\newpage

\tableofcontents

\newpage

\pagenumbering{arabic}

%%% The following structure is a suggestion
%%% If you prefer to use your own, you can change everything
\section{Introduction}
\section{Data and Feature Engineering}
\section{Methods and Model Architectures}
The increase in the share of renewable energy sources in total energy production leads to a increasingly fluctuating power generation. Therefore, measures for a stable energy supply will gain importance. Information about future energy production could enable power plant operators to control their power production up and down in order to match the incoming renewable energy. A future prediction could also help to store energy in a targeted manner. 
With this background, the goal of this work is to develop a machine learning model that predicts the renewable energy production 12 hours into the future.\\
There is a variety of model architectures for this problem, including CNNs (Convolutional Neural Networks), RNNs (Recurrent Neural Networks), Autoencoders, GRUs (Gated Recurrent Units) or LSTMs (Long Short Term Memory). This paper will first explore and evaluate existing models regarding their accuracy, then iteratively develop an advanced and optimized model architecture.\\

Description of the project. Images are always a plus. You should reference each figure in the text and explain what can be seen. The flow field in Figure~\ref{fig:flow} shows particle velocities. It is taken from~\cite{Author2020}. What is the problem? What is the goal? What is your idea?\\ 

\begin{figure}[h]
	\centering
	\includegraphics[scale=0.9]{Figures/example_cover.png}
	\caption{Flow field around a rectangular obstacle.}
	\label{fig:flow}
\end{figure}

%\lipsum[2]

You can include wrapped figures and tables, like Table~\ref{tab:features}. To make it work, there should be no newlines between the wrapped table or figure and the surrounding text. Normal tables work just as usual, like in Table~\ref{tab:other_parameters}.\\
%\lipsum[1]

\begin{wraptable}{r}{0.55\textwidth}
	%\begin{table}[]
	%\centering
	%\vspace{0pt}
	\begin{tabular}{@{} lrrrrrr @{}}
		\toprule
		 & $L$ & $\lambda^S$ & $\mu^S$ & $k_{0\text{S}}^{\text{F}1}$ & $k_{0\text{S}}^{\text{F}2}$ & $k_{0\text{S}}^{\text{F}3}$ \\
		\midrule
		min & 0.6 & 3.1 & 15.8 & 0.1 & 0.003 & 0.1 \\
		max & 8.3 & 196.8 & 285.5 & 1.0 & 0.1 & 0.9 \\
		example & 1.33 & 21 & 126 & 0.18 & 0.05 & 0.23 \\
		\bottomrule
	\end{tabular}
	\caption{Feature values used for training the CNN.}\label{tab:features}
\end{wraptable} 

%\lipsum[2-3]


\begin{table}[]
	\centering
	%\vspace{0pt}
	\begin{tabular}{@{} lrrrrrr @{}} 
		\toprule
		 & $L$ & $\lambda^S$ & $\mu^S$ & $k_{0\text{S}}^{\text{F}1}$ & $k_{0\text{S}}^{\text{F}2}$ & $k_{0\text{S}}^{\text{F}3}$ \\
		\midrule
		min & 0.6 & 3.1 & 15.8 & 0.1 & 0.003 & 0.1 \\
		max & 8.3 & 196.8 & 285.5 & 1.0 & 0.1 & 0.9 \\
		example & 1.33 & 21 & 126 & 0.18 & 0.05 & 0.23 \\
		\bottomrule
	\end{tabular}
	\caption{Values used to create the simulation in Figure~\ref{fig:flow}.}\label{tab:other_parameters}
\end{table}


\section{Results and Optimization}
\section{conclusion}
\section{outlook}

\subsection{The Navier-Stokes-Equations}

Equations as usual, like in Equation~\ref{equ:max_entropy}. Recall that equations like

\begin{equation}\label{equ:max_entropy}
	\phi_\mathrm{max} = \max_{\theta} \left[ \log(\sin(\theta - \exp(\theta))) - \theta^2 \right]
\end{equation}

are part of the text and should be treated like a word.

%\lipsum[1]

Or without numbering, like the realtivistic kinetic energy 

\begin{equation*}
	E = \frac{mc^2}{\sqrt{1 - \frac{v^2}{c^2}}},
\end{equation*}

which yields the Newtonian kinetic energy $E = \frac 1 2 m v^2$ when linearized for small velocities $v$.

\subsection{Solution Method}
%\lipsum[3-4]
\begin{wrapfigure}{r}{0.55\textwidth}
	\centering
	\includegraphics[scale=1.0]{Figures/example_cover.png}
	\caption{You can also use wrapped figures like this.}
	\label{fig:wrapfigure_example}
\end{wrapfigure}
%\lipsum[5] \\ \newline
Figure~\ref{fig:wrapfigure_example} shows how to include wrapped figures that text can float around. Depending on what is depicted and how large it is, it may look better than a full figure.

\section{Convolutional Neural Networks}

%\lipsum

\section{Idea and Data Generation}

%\lipsum

\section{Results}

%\lipsum

\begin{footnotesize}
\bibliographystyle{acm}


%%% This builds the bibliography
%%% You can use either bibtex with the following line and the "bibliography.bib" file
%%% OR ALTERNATIVELY comment the next line and uncomment the "thebibliography" environment

\bibliography{bibliography}

%%% comment the line above and uncomment the lines below
%%% for manually creating the bibliography

%\begin{thebibliography}{1}
%	
%	\bibitem{Author2020}% 
%	\textsc{A.\,W.~Harrow, A.~Hassidim, and S.~Lloyd},
%	\emph{Physical review letters 113} (15), 150502, (2009)
%	
%	\bibitem{Author2012}% 
%	\textsc{A.\,Y.~Kitaev},
%	\emph{Russian Mathematical Surveys 52} (6) p.\,1191--1249, (1997)
%	
%	
%\end{thebibliography}


\end{footnotesize}


%\newpage

\end{document}
